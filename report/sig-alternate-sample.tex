% This is "sig-alternate.tex" V2.1 April 2013
% This file should be compiled with V2.5 of "sig-alternate.cls" May 2012
%
% This example file demonstrates the use of the 'sig-alternate.cls'
% V2.5 LaTeX2e document class file. It is for those submitting
% articles to ACM Conference Proceedings WHO DO NOT WISH TO
% STRICTLY ADHERE TO THE SIGS (PUBS-BOARD-ENDORSED) STYLE.
% The 'sig-alternate.cls' file will produce a similar-looking,
% albeit, 'tighter' paper resulting in, invariably, fewer pages.
%
% ----------------------------------------------------------------------------------------------------------------
% This .tex file (and associated .cls V2.5) produces:
%       1) The Permission Statement
%       2) The Conference (location) Info information
%       3) The Copyright Line with ACM data
%       4) NO page numbers
%
% as against the acm_proc_article-sp.cls file which
% DOES NOT produce 1) thru' 3) above.
%
% Using 'sig-alternate.cls' you have control, however, from within
% the source .tex file, over both the CopyrightYear
% (defaulted to 200X) and the ACM Copyright Data
% (defaulted to X-XXXXX-XX-X/XX/XX).
% e.g.
% \CopyrightYear{2007} will cause 2007 to appear in the copyright line.
% \crdata{0-12345-67-8/90/12} will cause 0-12345-67-8/90/12 to appear in the copyright line.
%
% ---------------------------------------------------------------------------------------------------------------
% This .tex source is an example which *does* use
% the .bib file (from which the .bbl file % is produced).
% REMEMBER HOWEVER: After having produced the .bbl file,
% and prior to final submission, you *NEED* to 'insert'
% your .bbl file into your source .tex file so as to provide
% ONE 'self-contained' source file.
%
% ================= IF YOU HAVE QUESTIONS =======================
% Questions regarding the SIGS styles, SIGS policies and
% procedures, Conferences etc. should be sent to
% Adrienne Griscti (griscti@acm.org)
%
% Technical questions _only_ to
% Gerald Murray (murray@hq.acm.org)
% ===============================================================
%
% For tracking purposes - this is V2.0 - May 2012

\documentclass{sig-alternate-05-2015}


\begin{document}

% Copyright
%\setcopyright{acmcopyright}
%\setcopyright{acmlicensed}
%\setcopyright{rightsretained}
%\setcopyright{usgov}
%\setcopyright{usgovmixed}
%\setcopyright{cagov}
%\setcopyright{cagovmixed}


% DOI
%\doi{10.475/123_4}

% ISBN
%\isbn{123-4567-24-567/08/06}

%Conference
%\conferenceinfo{PLDI '13}{June 16--19, 2013, Seattle, WA, USA}

%\acmPrice{\$15.00}

%
% --- Author Metadata here ---
%\conferenceinfo{WOODSTOCK}{'97 El Paso, Texas USA}
%\CopyrightYear{2007} % Allows default copyright year (20XX) to be over-ridden - IF NEED BE.
%\crdata{0-12345-67-8/90/01}  % Allows default copyright data (0-89791-88-6/97/05) to be over-ridden - IF NEED BE.
% --- End of Author Metadata ---

\title{ML Exercise 2 - Group 19}
\subtitle{Classification Tasks}
%
% You need the command \numberofauthors to handle the 'placement
% and alignment' of the authors beneath the title.
%
% For aesthetic reasons, we recommend 'three authors at a time'
% i.e. three 'name/affiliation blocks' be placed beneath the title.
%
% NOTE: You are NOT restricted in how many 'rows' of
% "name/affiliations" may appear. We just ask that you restrict
% the number of 'columns' to three.
%
% Because of the available 'opening page real-estate'
% we ask you to refrain from putting more than six authors
% (two rows with three columns) beneath the article title.
% More than six makes the first-page appear very cluttered indeed.
%
% Use the \alignauthor commands to handle the names
% and affiliations for an 'aesthetic maximum' of six authors.
% Add names, affiliations, addresses for
% the seventh etc. author(s) as the argument for the
% \additionalauthors command.
% These 'additional authors' will be output/set for you
% without further effort on your part as the last section in
% the body of your article BEFORE References or any Appendices.

\numberofauthors{3} %  in this sample file, there are a *total*
% of EIGHT authors. SIX appear on the 'first-page' (for formatting
% reasons) and the remaining two appear in the \additionalauthors section.
%
\author{
% You can go ahead and credit any number of authors here,
% e.g. one 'row of three' or two rows (consisting of one row of three
% and a second row of one, two or three).
%
% The command \alignauthor (no curly braces needed) should
% precede each author name, affiliation/snail-mail address and
% e-mail address. Additionally, tag each line of
% affiliation/address with \affaddr, and tag the
% e-mail address with \email.
%
% 1st. author
\alignauthor
Lukas Stanek \\
       \affaddr{Technical University of Vienna}\\
       \affaddr{Karlsplatz 13}\\
       \affaddr{1040, Vienna}\\
       \email{----}
% 2nd. author
Thomas Appler \\
       \affaddr{Technical University of Vienna}\\
       \affaddr{Karlsplatz 13}\\
       \affaddr{1040, Vienna}\\
       \email{----}
% 3rd. author
\alignauthor
Marten Sigwart \\
       \affaddr{Technical University of Vienna}\\
       \affaddr{Karlsplatz 13}\\
       \affaddr{1040, Vienna}\\
       \email{e1638152@student.tuwien.ac.at}
}
% There's nothing stop
% Just remember to make sure that the TOTAL number of authors
% is the number that will appear on the first page PLUS the
% number that will appear in the \additionalauthors section.

\maketitle


\section{Introduction}
TODO\\


\subsection{4 data sets}
TODO\\

\subsubsection{Data Set 1: Letter Recognition}
TODO\\
\subsubsection{Data Set 2: Internet Advertisements}
TODO\\
\subsubsection{Data Set 3: Leukemia}
TODO\\
\subsubsection{Data Set 4: Congressional Voting}
TODO\\


\subsection{4 classification algorithms}
TODO\\

\subsubsection{Algorithm 1: K-Nearest Neighbours}
TODO\\
\subsubsection{Algorithm 2: }
TODO\\
\subsubsection{Algorithm 3: }
TODO\\
\subsubsection{Algorithm 4:}
TODO\\


% Section Data Set 1
\section{Data Set 1: Letter Recognition}
The objective for this data set is to predict a letter shown on a rectangular, black and white, display. There are 16 numerical features  provided for determining the correct letter. These attributes provide information about statistical properties of the letter like total number of pixels, width, mean of x-axis pixels, etc.
All of these values were then scaled into a range from 0 through 15. The 26 capital letters to predict in this set were taken from 20 different fonts. The letters are more or less evenly spread over the 20.000 instances, within a range of 734 to 813 occurrences. This data set contains no missing values. For comparable results 10-fold cross validation was used to measure the algorithms precision.
As for performance criteria with this data set only the percentage of correctly classified instances will be considered. The reason for this being that if a letter is wrongly classified, it doesn't matter which other letter was recognized instead of the correct one. 
\\
\subsection{Preprocessing}
As the data set was only provided as CSV file without header row, we added a header row for displaying the attribute names in Weka and therefore facilitating the analysis of the results. 
\\
\subsection{K-Nearest Neighbors}
In order to be able to compare the results we always used the same number of neighbors: 1,2,3,4,5,6,8,10,16,20,40,100.
\subsubsection{Default implementation}
At first we started out using the default KNN implementation of Weka with default settings. We started out with increasing the amount of neighbors starting at one. Though with this settings we got the best results using just the nearest neighbor. \\\\
\begin{tabular}{ l | c |c }
K & \% Correct & \% Wrong \\
1 & 95.96 & 4.04 \\
2 & 94.925 & 5.075 \\
3 & 95.635 & 4.365 \\
\end{tabular}
\\
\subsection{Algorithm 2}
TODO\\
\subsection{Algorithm 3}
TODO\\
\subsection{Algorithm 4}
TODO\\
\subsection{Conclusion}
TODO\\


% Section Data Set 2
\section{Data Set 2: Internet Advertisements}
TODO\\
\subsection{Preprocessing}
TODO what preprocessing was done\\
\subsection{K-Nearest Neighbours}
TODO\\
\subsection{Algorithm 2}
TODO\\
\subsection{Algorithm 3}
TODO\\
\subsection{Algorithm 4}
TODO\\
\subsection{Conclusion}
TODO\\


% Section Data Set 3
\section{Data Set 3: Leukemia}
TODO\\
\subsubsection{Preprocessing}
TODO what preprocessing was done\\
\subsubsection{K-Nearest Neighbours}
TODO\\
\subsubsection{Algorithm 2}
TODO\\
\subsubsection{Algorithm 3}
TODO\\
\subsubsection{Algorithm 4}
TODO\\
\subsubsection{Conclusion}
TODO\\


% Section Data Set 4
\section{Data Set 4: Congressional Voting}
TODO\\
\subsubsection{Preprocessing}
TODO what preprocessing was done\\
\subsubsection{K-Nearest Neighbours}
TODO\\
\subsubsection{Algorithm 2}
TODO\\
\subsubsection{Algorithm 3}
TODO\\
\subsubsection{Algorithm 4}
TODO\\
\subsubsection{Conclusion}
TODO\\


\section{Conclusion}
TODO\\



\end{document}
